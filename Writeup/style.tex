\documentclass[runningheads]{llncs}

\newif\ifLNCSVER
%\LNCSVERtrue
\LNCSVERfalse

\ifLNCSVER

\else
  \usepackage[letterpaper,hmargin=1.25in,vmargin=1.0in]{geometry}
\fi



%\usepackage[letterpaper,hmargin=1.25in,vmargin=1.25in]{geometry}
\setcounter{tocdepth}{3}
\usepackage{graphicx}
\usepackage{amsmath,url,bm,amssymb,latexsym,multirow,multicol,xspace,subfigure,afterpage,lscape,booktabs,cancel,dashbox}
\usepackage{algorithm,algpseudocode}
\usepackage{colortbl}
\usepackage{comment,multirow}
\usepackage[normalem]{ulem}
\usepackage{hyperref}
\usepackage{todonotes}
\usepackage{tikz,pgfplots}
\usepackage{subfigure}
\usepackage{autobreak}
\usepackage{mathtools}


\renewcommand{\floatpagefraction}{0.95}
\renewcommand{\topfraction}{0.95}

\usepackage{lineno}

\usepackage{listings}
\lstset{
	backgroundcolor=\color{white},   % choose the background color; you must add \usepackage{color} or \usepackage{xcolor}; should come as last argument
	basicstyle=\ttfamily\scriptsize,        % the size of the fonts that are used for the code
	breakatwhitespace=false,         % sets if automatic breaks should only happen at whitespace
	breaklines=true,                 % sets automatic line breaking
	captionpos=b,                    % sets the caption-position to bottom
	commentstyle=\color{green},    % comment style
	deletekeywords={...},            % if you want to delete keywords from the given language
	escapeinside={\%*}{*)},          % if you want to add LaTeX within your code
	extendedchars=true,              % lets you use non-ASCII characters; for 8-bits encodings only, does not work with UTF-8
	firstnumber=1,                % start line enumeration with line 1000
	frame=single,	                   % adds a frame around the code
	keepspaces=true,                 % keeps spaces in text, useful for keeping indentation of code (possibly needs columns=flexible)
	keywordstyle=\color{blue},       % keyword style
	language=Python,                 % the language of the code
	morekeywords={*,...},            % if you want to add more keywords to the set
	numbers=left,                    % where to put the line-numbers; possible values are (none, left, right)
	numbersep=5pt,                   % how far the line-numbers are from the code
	numberstyle=\tiny\color{gray}, % the style that is used for the line-numbers
	rulecolor=\color{black},         % if not set, the frame-color may be changed on line-breaks within not-black text (e.g. comments (green here))
	showspaces=false,                % show spaces everywhere adding particular underscores; it overrides 'showstringspaces'
	showstringspaces=false,          % underline spaces within strings only
	showtabs=false,                  % show tabs within strings adding particular underscores
	stepnumber=1,                    % the step between two line-numbers. If it's 1, each line will be numbered
	%stringstyle=\color{mauve},     % string literal style
	tabsize=2,	                   % sets default tabsize to 2 spaces
	title=\lstname                   % show the filename of files included with \lstinputlisting; also try caption instead of title
}

\renewcommand{\floatpagefraction}{0.95}
\renewcommand{\topfraction}{0.95}

\makeatletter
\def\add(#1,#2){{%
		\newcount\@a \@a = #1 \relax
		\advance \@a by #2 \relax
		\the\@a
}}
\makeatother

%\newcommand*\patchAmsMathEnvironmentForLineno[1]{%
%	\expandafter\let\csname old#1\expandafter\endcsname\csname #1\endcsname
%	\expandafter\let\csname oldend#1\expandafter\endcsname\csname end#1\endcsname
%	\renewenvironment{#1}%
%	{\linenomath\csname old#1\endcsname}%
%	{\csname oldend#1\endcsname\endlinenomath}}%
%\newcommand*\patchBothAmsMathEnvironmentsForLineno[1]{%
%	\patchAmsMathEnvironmentForLineno{#1}%
%	\patchAmsMathEnvironmentForLineno{#1*}}%
%\AtBeginDocument{%
%	\patchBothAmsMathEnvironmentsForLineno{equation}%
%	\patchBothAmsMathEnvironmentsForLineno{align}%
%	\patchBothAmsMathEnvironmentsForLineno{flalign}%
%	\patchBothAmsMathEnvironmentsForLineno{alignat}%
%	\patchBothAmsMathEnvironmentsForLineno{gather}%
%	\patchBothAmsMathEnvironmentsForLineno{multline}%
%}
%\linenumbers

\usepackage{color}
\newcommand{\yt}[1]{\textcolor{red}{[{\bf Yosuke:} #1]}}
\newcommand{\yh}[1]{\textcolor{red}{[{Yonglin:} #1]}}
\newcommand{\qw}[1]{\textcolor{red}{[{Qingju:} #1]}}
%\newcommand{\discuss}[1]{\textcolor{red}{#1}}
\newcommand{\rc}[1]{\textcolor{red}{#1}}

\algdef{SE}[DOWHILE]{Do}{doWhile}{\algorithmicdo}[1]{\algorithmicwhile\ #1}

%\newtheorem{example}{Example}
\newtheorem{assumption}{Assumption}
\newtheorem{hypothesis}{Hypothesis}
\newtheorem{statement}{Statement}
%\newtheorem{question}{Open Question}

%
\newcommand{\gitaddress}{\url{https://github.com/ysktodo/milp-three-subset-wo-unknown}\xspace}

%
\newcommand{\seti}{\mathbb{X}}
\newcommand{\seto}{\mathbb{Y}}

%newcommand
%\newcommand{\A}{{\cal A}}
%\newcommand{\B}{{\cal B}}
%\newcommand{\C}{{\cal C}}
\newcommand{\D}{{\cal D}}
\newcommand{\T}{{\cal T}}

\DeclareMathOperator{\Prob}{Pr}

\newcommand{\F}{\mathbb{F}}
%\newcommand{\U}{{\cal U}}
\newcommand{\Sbox}{S-box\xspace}
\newcommand{\Sboxes}{S-boxes\xspace}
\newcommand{\etal}{{et al.}\xspace}
\newcommand{\eg}{{e.g.}\xspace}
\newcommand{\ie}{{i.e.}\xspace}
\newcommand{\st}{{s.t.}\xspace}

\newcommand{\ACORN}{\textsc{ACORN}\xspace}
\newcommand{\LUFFA}{{\it Luffa}\xspace}
\newcommand{\PHOTON}{\texttt{PHOTON}\xspace}
\newcommand{\LED}{\texttt{LED}\xspace}
\newcommand{\KECCAK}{\textsc{Keccak}\xspace}
\newcommand{\SQUARE}{\textsc{Square}\xspace}
\newcommand{\ANUBIS}{\textsc{Anubis}\xspace}
\newcommand{\WHIRLPOOL}{\textsc{Whirlpool}\xspace}
\newcommand{\GROESTL}{Gr{\o}stl\xspace}
\newcommand{\NOEKEON}{\textsc{Noekeon}\xspace}
\newcommand{\SKINNY}{\texttt{SKINNY}\xspace}
\newcommand{\MYSTERION}{\textsf{Mysterion}\xspace}
\newcommand{\lilliput}{\textsc{Lilliput}\xspace}
\newcommand{\SPARX}{\textsc{Sparx}\xspace}

\newcommand{\SIMON}{\textsc{Simon}}
\newcommand{\SIMECK}{\textsf{Simeck}}

\newcommand{\TRIVIUM}{\textsc{Trivium}\xspace}


\newcommand*{\NL}[1]{( b_{t+\add(#1,3)}b_{t+\add(#1,67)} + b_{t+\add(#1,11)}b_{t+\add(#1,13)} + b_{t+\add(#1,17)}b_{t+\add(#1,18)} + b_{t+\add(#1,27)}b_{t+\add(#1,59)} + b_{t+\add(#1,40)}b_{t+\add(#1,48)} + b_{t+\add(#1,61)}b_{t+\add(#1,65)} + b_{t+\add(#1,68)}b_{t+\add(#1,84)} \\ & \quad+ b_{t+\add(#1,88)}b_{t+\add(#1,92)}b_{t+\add(#1,93)}b_{t+\add(#1,95)} + b_{t+\add(#1,22)}b_{t+\add(#1,24)}b_{t+\add(#1,25)} + b_{t+\add(#1,70)}b_{t+\add(#1,78)}b_{t+\add(#1,82)} )}

\newcommand{\grain}{Grain-128\xspace}
\newcommand{\graina}{Grain-128a\xspace}
\newcommand{\grainv}{Grain-v1\xspace}
\newcommand*{\ls}[1]{L^{(#1)}}
\newcommand*{\lsp}[1]{s'^{(#1)}}
\newcommand*{\ns}[1]{N^{(#1)}}
\newcommand{\tz}{\mathbb{T}_z}
\newcommand{\tb}{\mathbb{T}_b}
\newcommand*{\mask}[1]{\Lambda_{#1}}
\newcommand{\set}{\mathbb{S}}

\newcommand{\GF}{\mathrm{GF}}
\newcommand{\mat}{F}
\newcommand*{\A}[1]{A_{#1}}
\newcommand*{\trans}[1]{{}^\mathrm{T}\!{#1}}

\newcommand*{\dl}[1]{\cancel{#1}}
