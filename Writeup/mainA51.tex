%style input
\documentclass[runningheads]{llncs}

\newif\ifLNCSVER
%\LNCSVERtrue
\LNCSVERfalse

\ifLNCSVER

\else
  \usepackage[letterpaper,hmargin=1.25in,vmargin=1.0in]{geometry}
\fi



%\usepackage[letterpaper,hmargin=1.25in,vmargin=1.25in]{geometry}
\setcounter{tocdepth}{3}
\usepackage{graphicx}
\usepackage{amsmath,url,bm,amssymb,latexsym,multirow,multicol,xspace,subfigure,afterpage,lscape,booktabs,cancel,dashbox}
\usepackage{algorithm,algpseudocode}
\usepackage{colortbl}
\usepackage{comment,multirow}
\usepackage[normalem]{ulem}
\usepackage{hyperref}
\usepackage{todonotes}
\usepackage{tikz,pgfplots}
\usepackage{subfigure}
\usepackage{autobreak}
\usepackage{mathtools}


\renewcommand{\floatpagefraction}{0.95}
\renewcommand{\topfraction}{0.95}

\usepackage{lineno}

\usepackage{listings}
\lstset{
	backgroundcolor=\color{white},   % choose the background color; you must add \usepackage{color} or \usepackage{xcolor}; should come as last argument
	basicstyle=\ttfamily\scriptsize,        % the size of the fonts that are used for the code
	breakatwhitespace=false,         % sets if automatic breaks should only happen at whitespace
	breaklines=true,                 % sets automatic line breaking
	captionpos=b,                    % sets the caption-position to bottom
	commentstyle=\color{green},    % comment style
	deletekeywords={...},            % if you want to delete keywords from the given language
	escapeinside={\%*}{*)},          % if you want to add LaTeX within your code
	extendedchars=true,              % lets you use non-ASCII characters; for 8-bits encodings only, does not work with UTF-8
	firstnumber=1,                % start line enumeration with line 1000
	frame=single,	                   % adds a frame around the code
	keepspaces=true,                 % keeps spaces in text, useful for keeping indentation of code (possibly needs columns=flexible)
	keywordstyle=\color{blue},       % keyword style
	language=Python,                 % the language of the code
	morekeywords={*,...},            % if you want to add more keywords to the set
	numbers=left,                    % where to put the line-numbers; possible values are (none, left, right)
	numbersep=5pt,                   % how far the line-numbers are from the code
	numberstyle=\tiny\color{gray}, % the style that is used for the line-numbers
	rulecolor=\color{black},         % if not set, the frame-color may be changed on line-breaks within not-black text (e.g. comments (green here))
	showspaces=false,                % show spaces everywhere adding particular underscores; it overrides 'showstringspaces'
	showstringspaces=false,          % underline spaces within strings only
	showtabs=false,                  % show tabs within strings adding particular underscores
	stepnumber=1,                    % the step between two line-numbers. If it's 1, each line will be numbered
	%stringstyle=\color{mauve},     % string literal style
	tabsize=2,	                   % sets default tabsize to 2 spaces
	title=\lstname                   % show the filename of files included with \lstinputlisting; also try caption instead of title
}

\renewcommand{\floatpagefraction}{0.95}
\renewcommand{\topfraction}{0.95}

\makeatletter
\def\add(#1,#2){{%
		\newcount\@a \@a = #1 \relax
		\advance \@a by #2 \relax
		\the\@a
}}
\makeatother

%\newcommand*\patchAmsMathEnvironmentForLineno[1]{%
%	\expandafter\let\csname old#1\expandafter\endcsname\csname #1\endcsname
%	\expandafter\let\csname oldend#1\expandafter\endcsname\csname end#1\endcsname
%	\renewenvironment{#1}%
%	{\linenomath\csname old#1\endcsname}%
%	{\csname oldend#1\endcsname\endlinenomath}}%
%\newcommand*\patchBothAmsMathEnvironmentsForLineno[1]{%
%	\patchAmsMathEnvironmentForLineno{#1}%
%	\patchAmsMathEnvironmentForLineno{#1*}}%
%\AtBeginDocument{%
%	\patchBothAmsMathEnvironmentsForLineno{equation}%
%	\patchBothAmsMathEnvironmentsForLineno{align}%
%	\patchBothAmsMathEnvironmentsForLineno{flalign}%
%	\patchBothAmsMathEnvironmentsForLineno{alignat}%
%	\patchBothAmsMathEnvironmentsForLineno{gather}%
%	\patchBothAmsMathEnvironmentsForLineno{multline}%
%}
%\linenumbers

\usepackage{color}
\newcommand{\yt}[1]{\textcolor{red}{[{\bf Yosuke:} #1]}}
\newcommand{\yh}[1]{\textcolor{red}{[{Yonglin:} #1]}}
\newcommand{\qw}[1]{\textcolor{red}{[{Qingju:} #1]}}
%\newcommand{\discuss}[1]{\textcolor{red}{#1}}
\newcommand{\rc}[1]{\textcolor{red}{#1}}

\algdef{SE}[DOWHILE]{Do}{doWhile}{\algorithmicdo}[1]{\algorithmicwhile\ #1}

%\newtheorem{example}{Example}
\newtheorem{assumption}{Assumption}
\newtheorem{hypothesis}{Hypothesis}
\newtheorem{statement}{Statement}
%\newtheorem{question}{Open Question}

%
\newcommand{\gitaddress}{\url{https://github.com/ysktodo/milp-three-subset-wo-unknown}\xspace}

%
\newcommand{\seti}{\mathbb{X}}
\newcommand{\seto}{\mathbb{Y}}

%newcommand
%\newcommand{\A}{{\cal A}}
%\newcommand{\B}{{\cal B}}
%\newcommand{\C}{{\cal C}}
\newcommand{\D}{{\cal D}}
\newcommand{\T}{{\cal T}}

\DeclareMathOperator{\Prob}{Pr}

\newcommand{\F}{\mathbb{F}}
%\newcommand{\U}{{\cal U}}
\newcommand{\Sbox}{S-box\xspace}
\newcommand{\Sboxes}{S-boxes\xspace}
\newcommand{\etal}{{et al.}\xspace}
\newcommand{\eg}{{e.g.}\xspace}
\newcommand{\ie}{{i.e.}\xspace}
\newcommand{\st}{{s.t.}\xspace}

\newcommand{\ACORN}{\textsc{ACORN}\xspace}
\newcommand{\LUFFA}{{\it Luffa}\xspace}
\newcommand{\PHOTON}{\texttt{PHOTON}\xspace}
\newcommand{\LED}{\texttt{LED}\xspace}
\newcommand{\KECCAK}{\textsc{Keccak}\xspace}
\newcommand{\SQUARE}{\textsc{Square}\xspace}
\newcommand{\ANUBIS}{\textsc{Anubis}\xspace}
\newcommand{\WHIRLPOOL}{\textsc{Whirlpool}\xspace}
\newcommand{\GROESTL}{Gr{\o}stl\xspace}
\newcommand{\NOEKEON}{\textsc{Noekeon}\xspace}
\newcommand{\SKINNY}{\texttt{SKINNY}\xspace}
\newcommand{\MYSTERION}{\textsf{Mysterion}\xspace}
\newcommand{\lilliput}{\textsc{Lilliput}\xspace}
\newcommand{\SPARX}{\textsc{Sparx}\xspace}

\newcommand{\SIMON}{\textsc{Simon}}
\newcommand{\SIMECK}{\textsf{Simeck}}

\newcommand{\TRIVIUM}{\textsc{Trivium}\xspace}


\newcommand*{\NL}[1]{( b_{t+\add(#1,3)}b_{t+\add(#1,67)} + b_{t+\add(#1,11)}b_{t+\add(#1,13)} + b_{t+\add(#1,17)}b_{t+\add(#1,18)} + b_{t+\add(#1,27)}b_{t+\add(#1,59)} + b_{t+\add(#1,40)}b_{t+\add(#1,48)} + b_{t+\add(#1,61)}b_{t+\add(#1,65)} + b_{t+\add(#1,68)}b_{t+\add(#1,84)} \\ & \quad+ b_{t+\add(#1,88)}b_{t+\add(#1,92)}b_{t+\add(#1,93)}b_{t+\add(#1,95)} + b_{t+\add(#1,22)}b_{t+\add(#1,24)}b_{t+\add(#1,25)} + b_{t+\add(#1,70)}b_{t+\add(#1,78)}b_{t+\add(#1,82)} )}

\newcommand{\grain}{Grain-128\xspace}
\newcommand{\graina}{Grain-128a\xspace}
\newcommand{\grainv}{Grain-v1\xspace}
\newcommand*{\ls}[1]{L^{(#1)}}
\newcommand*{\lsp}[1]{s'^{(#1)}}
\newcommand*{\ns}[1]{N^{(#1)}}
\newcommand{\tz}{\mathbb{T}_z}
\newcommand{\tb}{\mathbb{T}_b}
\newcommand*{\mask}[1]{\Lambda_{#1}}
\newcommand{\set}{\mathbb{S}}

\newcommand{\GF}{\mathrm{GF}}
\newcommand{\mat}{F}
\newcommand*{\A}[1]{A_{#1}}
\newcommand*{\trans}[1]{{}^\mathrm{T}\!{#1}}

\newcommand*{\dl}[1]{\cancel{#1}}

%\usepackage{setspace}
%\setstretch{2}

        \begin{document}
% start of an individual contribution
\mainmatter
% first the title is needed
\title{Analyze the Tradeoffs in A5/1 Key-Recovery Attacks}
%\author{Yonglin Hao\inst{1}}
%\institute{
%	State Key Laboratory of Cryptology, P.O. Box 5159, Beijing 100878, China, \email{haoyonglin@yeah.net} \and
%	Ruhr University Bochum, Horst G{\"o}rtz Institute for IT Security, Germany, \email{gregor.leander@rub.de} \and
%	FHNW, Windisch, Switzerland, \email{willimeier48@gmail.com} \and
%	NTT Secure Platform Laboratories, Tokyo 180-8585, Japan, \email{yosuke.todo.xt@hco.ntt.co.jp} \and
%	SnT, University of Luxembourg, Esch-sur-Alzette, Luxembourg, \email{qingju.wang@uni.lu}}
\maketitle


%\input{introduction_abst}

%\pagewiselinenumbers



%%%%%%%%%%%%%%%%%%%%%%%%%%%%%%%%%%%%%%%%%%%%%%%%%%%%%%%%%%%%%%%%%%%%%%%%%%%%%%%%%%%%%%%%%
%%%%%%%%%%%%%%%%%%%%%%%%%%%%%%%%%%%%%   Abstract   %%%%%%%%%%%%%%%%%%%%%%%%%%%%%%%%%%%%%%
%%%%%%%%%%%%%%%%%%%%%%%%%%%%%%%%%%%%%%%%%%%%%%%%%%%%%%%%%%%%%%%%%%%%%%%%%%%%%%%%%%%%%%%%%
\begin{abstract}
We analyze the
\keywords{stream ciphers, A5/1, guess-and-determine}
\end{abstract}

\section{Introduction}
We analyze the attack given by Bin Zhang in \cite{AC:Zhang19} only to find that the complexities given in \cite{AC:Zhang19} are wrong.
As a remedy, we give detailed analysis to all the possible tradeoffs in the guess-and-determine attacks on A5/1 and give the best practical attack requiring simple $2^{44.60}$ computations and a negligible memory complexity.


\section{Preliminary}
A5/1 has a 64-bit internal state consisting of 3 registers of sizes 19, 22, 23 respectively.
We denote the 64-bit state at time $t$ ($t=0,1,2,\ldots$) as
\begin{equation}\label{eq:StateAndRi}
\begin{split}
   S^t= & (R1^t, R2^t, R3^t)\\
     =& (S^t[0,\ldots, 18],S^t[19,\ldots, 40],S^t[41,\ldots, 63])\\
     =&(R1^t[0,\ldots,18],R2^t[0,\ldots, 21],R3^t[0,\ldots,22])
\end{split}
\end{equation}
Before generating the output bit $z_t$, A5/1 round function will update the internal state $S^t\Rightarrow S^{t+1}$ in a stop-and-go manner as follows:
\begin{enumerate}
  \item Compute $maj_t$ as
\begin{equation}\label{eq:Majt}
  maj_t=(R1^t[8]\cdot R2^t[10])\oplus (R1^t[8]\cdot R3^t[10])\oplus (R2^t[10]\cdot R3^t[10])
\end{equation}
  \item If $R1^t[8]\neq maj_t$, $R1^{t+1}\leftarrow R1^t$, otherwise, call $\tt{updateR1}$ as follows:
  \begin{equation}\label{eq:UpdateR1}
    R1^{t+1}[i]\leftarrow\left\{
    \begin{aligned}
      &R1^t[i-1]\quad i\in [1,18]\\
      &R1^t[18]\oplus R1^t[17]\oplus R1^t[16]\oplus R1^t[13]
    \end{aligned}
    \right.
  \end{equation}
  \item If $R2^t[10]\neq maj_t$, $R2^{t+1}\leftarrow R2^t$, otherwise, call $\tt{updateR2}$ as follows:
    \begin{equation}\label{eq:UpdateR2}
    R2^{t+1}[i]\leftarrow\left\{
    \begin{aligned}
      &R2^t[i-1]\quad i\in [1,21]\\
      &R2^t[21]\oplus R2^t[20]
    \end{aligned}
    \right.
  \end{equation}
    \item If $R3^t[10]\neq maj_t$, $R3^{t+1}\leftarrow R3^t$, otherwise, call $\tt{updateR3}$ as follows:
    \begin{equation}\label{eq:UpdateR3}
    R3^{t+1}[i]\leftarrow\left\{
    \begin{aligned}
      &R3^t[i-1]\quad i\in [1,22]\\
      &R3^t[22]\oplus R3^t[21]\oplus R3^t[20]
    \end{aligned}
    \right.
  \end{equation}
\end{enumerate}
Then, the output keystream bit $z^t$ is generated as
\begin{equation}\label{eq:OutputZ}
  z^t=R1^{t+1}[18]\oplus R2^{t+1}[21]\oplus R3^{t+1}[22]
\end{equation}
\section{Move Patterns and Their Bit Condition Representations}
For each step $t=0,1,2\ldots$, whether the registers $R1,R2,R3$ are updated or not depends on the three control bits $(R1^t[8],R2^t[10],R3^t[10])$. 
For all $2^3$ $(R1^t[8],R2^t[10],R3^t[10])$ values of, there are 4 possible movements. 
Each movement corresponds to 2 $(R1^t[8],R2^t[10],R3^t[10])$ values and can also be represented as linear equations of state bits, also referred as bit conditions. 
The 4 movements are denoted as Move 0-4 as follows:
\begin{description}
  \item[Move 0] $\tt{updateR1}$, $\tt{updateR2}$ and $\tt{updateR3}$ are all called. 
  The corresponding $(R1^t[8],R2^t[10],R3^t[10])$ values are $(0,0,0)$ and $(1,1,1)$.
  The bit conditions are:
  \begin{equation}\label{eq:Move0BitCondition}
    \left\{
    \begin{aligned}
    &R1^t[8]=R2^t[10]\\
    &R1^t[8]=R3^t[10]
    \end{aligned}
    \right.
    \Leftrightarrow
    \left\{
    \begin{aligned}
    &S^t[8]\oplus S^t[29]=0\\
    &S^t[8]\oplus S^t[51]=0
    \end{aligned}
    \right.
  \end{equation}
  \item[Move 1] Only $\tt{updateR2}$ and $\tt{updateR3}$ are called. ($(R1^t[8],R2^t[10],R3^t[10])\in \{(0,1,1),(1,0,0)\}$)
  The corresponding $(R1^t[8],R2^t[10],R3^t[10])$ values are $(0,1,1)$ and $(1,0,0)$.
  The bit conditions are:
  \begin{equation}\label{eq:Move1BitCondition}
    \left\{
    \begin{aligned}
    &R1^t[8]=R2^t[10]\oplus 1\\
    &R1^t[8]=R3^t[10]\oplus 1
    \end{aligned}
    \right.
    \Leftrightarrow
    \left\{
    \begin{aligned}
    &S^t[8] \oplus S^t[29]=1\\
    &S^t[8] \oplus S^t[51]=1
    \end{aligned}
    \right.
  \end{equation}
  \item[Move 2] Only $\tt{updateR1}$ and $\tt{updateR3}$ are called. ($(R1^t[8],R2^t[10],R3^t[10])\in \{(0,1,1),(1,0,0)\}$)
  The corresponding $(R1^t[8],R2^t[10],R3^t[10])$ values are $(1,0,1)$ and $(0,1,0)$.
  The bit conditions are:
  \begin{equation}\label{eq:Move2BitCondition}
    \left\{
    \begin{aligned}
    &R1^t[8]=R2^t[10]\oplus 1\\
    &R1^t[8]=R3^t[10]
    \end{aligned}
    \right.
    \Leftrightarrow
    \left\{
    \begin{aligned}
    &S^t[8]\oplus S^t[29]=1\\
    &S^t[8]\oplus S^t[51]=0
    \end{aligned}
    \right.
  \end{equation}
  \item[Move 3] Only $\tt{updateR1}$ and $\tt{updateR2}$ are called. ($(R1^t[8],R2^t[10],R3^t[10])\in \{(0,1,1),(1,0,0)\}$)
  The corresponding $(R1^t[8],R2^t[10],R3^t[10])$ values are $(1,1,0)$ and $(0,0,1)$.
  The bit conditions are:
  \begin{equation}\label{eq:Move3BitCondition}
    \left\{
    \begin{aligned}
    &R1^t[8]=R2^t[10]\\
    &R1^t[8]=R3^t[10]\oplus 1
    \end{aligned}
    \right.
    \Leftrightarrow
    \left\{
    \begin{aligned}
    &S^t[8]\oplus S^t[29]=0\\
    &S^t[8]\oplus S^t[51]=1
    \end{aligned}
    \right.
  \end{equation}
\end{description}
We denote the movement $S^t\rightarrow S^{t+1}$ as $m^t\in \mathbb{F}_2^2= \{0,1,2,3\}$. 

\section{Move Guesses and Bit Condition Deductions}
In our guess and determine attack, for $t=0,1,...$, we first guess the movement corresponding to $S^t\rightarrow S^{t+1}$ and maintains a bit condition set $\mathcal{B}$ by adding new bit conditions corresponding to the new movement and the output $z^t$. 
For each step $t$, there are 3 bit conditions: 2 are from one of \eqref{eq:Move0BitCondition}, \eqref{eq:Move1BitCondition}, \eqref{eq:Move2BitCondition}, \eqref{eq:Move3BitCondition} according to the move guess and the rest is from the output $z^t$ as
\begin{equation}\label{eq:OutputBitCondition}
S^{t+1}[18]\oplus S^{t+1}[40]\oplus S^{t+1}[63]=z^t
\end{equation}
So each move guess can deduce 3 bit conditions. 
As can be seen, the move conditions \eqref{eq:Move0BitCondition}, \eqref{eq:Move1BitCondition}, \eqref{eq:Move2BitCondition}, \eqref{eq:Move3BitCondition} and the output condition \eqref{eq:OutputBitCondition} correspond to the internal state at different time instance. 
But our attack is targeted to recovering the initial state $S^0$. 
Therefore, we need to represent the internal states at different time instance $t$ with the $S^0$ so that the bit conditions are represented by $S^0$ bits as well. 
With the movements $m^0,\ldots, m^t$ guessed, each bits of $S^{t+1}$ are linear equations of $S^0$. 
We can track the equations with a 64-bit word.  


\section{Conclusion}





\ifLNCSVER
  \bibliographystyle{splncs}
\else
  \bibliographystyle{alpha}
\fi
\bibliography{bib/abbrev3,bib/crypto,myrefs}




\ifLNCSVER

\else

\fi



\end{document}

